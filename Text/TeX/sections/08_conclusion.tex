\subsection{Major Findings}
Technological advances and the resulting increasing availability of large samples of GPS movement data enabled the development of the algorithm presented in this work.

An algorithm was created that enables the analysis of the red kite's breeding cycle in several steps. In one run, for an individual or an entire cohort, the home range can be detected, the nest can be localized, and the breeding behaviour can be characterised on a daily basis. The modularity of the algorithm allows the analysis of individual aspects of the breeding cycle separately, which enables a broad range of applications. For the home range detection and nest detection steps, the thresholds are adjustable to suit the movement dimensions of the subpopulation under study. The algorithm for the analysis of the brood cycle, which produces day-by-day predictions for the respective brood phase, has the advantage that no prior knowledge of the extent of the movement patterns of the subpopulation under study is necessary, as the variables are all scaled and centred.

Predicting brood phases based on GPS movement data allows the brood cycles of several red kites to be studied remotely. This can help to significantly reduce the personnel and time required for field work and can generate valuable insights into the breeding behaviour of red kites, helping to better understand population dynamics and optimise conservation strategies. This can be helpful if monitoring on site is logistically not feasible, or if breeding behaviour is to be analysed on a large scale, for example with the LIFE EUROKITE dataset \parencite{lifeeurokiteproject}, which contains red kite movement data across Europe.

\subsection{Limitations}
In the two steps of home range and nest detection, prior knowledge of the population under study is required to set thresholds in order to obtain reasonable results. Although it can be assumed in ornithological research that this prior knowledge is available in most cases, it would be a simplification of the process if data could be analysed without parameterisation of the algorithm and the thresholds were set automatically on the basis of the movement data.

Many more variants of an MLRM could have been created and compared to find an optimised model. For example, the variables could have been weighted according to their estimates to increase the significance of the predictions. Also in the selection of variables, many more variables could have been tested, which would have resulted in a different performance of the model. A different parameterisation of the MLRM could possibly lead to better results.

Despite the good performance of the MLRM, it was not possible to develop an approach that could reliably interpret the duration of the two brood phases and conclude the brood success from them. A classification based solely on a deterministic approach does not seem to allow a reliable determination of brood success.

To enable the applicability of the algorithm to a broad number of datasets, this work was carried out with GPS movement data with a temporal resolution of one hour. However, fine-scale movement patterns may not be captured with this coarse resolution. The performance of the algorithm could possibly be improved by using data with a finer temporal resolution. However, this would increase the computational effort. It should also be noted that the higher resolution leads to higher spatio-temporal autocorrelation, which complicates the calculation of home ranges. Ultimately, such high-resolution data are rarer than data with a temporal resolution of one hour.

\subsection{Future Research}
In further steps, a more reliable classification of the model predictions into successful and unsuccessful broods would certainly be necessary. Instead of a deterministic approach based solely on biological rules, a probabilistic approach could achieve better results. Since the predictions are already calculated in probabilities, it would be optimal to derive further classification attempts of the individual days from these probabilities, instead of assigning the day to the category that achieves the highest probability. Another point that could improve the performance of the brood success classification would be an approach that determines the duration of each phase instead of the sum of its days. This could lead to better results especially because the duration of the phases was mostly underestimated by the approach presented in this work.
Improved interpretability of MLRM predictions could also lead to a more reliable definition of the timing of each brood phase. Breeding-relevant events such as egg laying or hatching could then be derived from this. With an improved classification method, even brood losses could be detected.
It would also be interesting to analyse the applicability of the algorithm to other species that show similar breeding behaviour. It may be possible to identify how the algorithm can be parameterised differently and gain insights into the effects of certain movement-specific variables on certain species.

These gains in knowledge about the remote interpretation of breeding behaviour allow the analysis of broods of numerous individuals even from areas that are difficult to access. This in turn contributes to the adaptation of conservation measures in order to protect endangered species.