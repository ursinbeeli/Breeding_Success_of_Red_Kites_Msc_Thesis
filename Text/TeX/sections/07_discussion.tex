This study showed that GPS movement data of red kites can be successfully used to analyse different aspects of the breeding cycle. With an accuracy of 77\%, it could be reliably determined whether an individual occupies a home range. In a further step, nest locations of red kites could be reliably localised with an accuracy of 88\%. For breeding birds, depending on the approach, it was possible to correctly predict whether an individual was breeding in up to 76\% of the days. However, a reliable classification of whether a breeding cycle was successful or not could not be achieved.

\subsection{Home Range Detection}
With an accuracy of 77\%, the overall proportion of correctly predicted cases is high. The high recall of 97\% is achieved by the fact that only 2.7\% (12 of 441 cases) of home ranges remained undetected. The precision of 69\% and the MCC of 60\% result from the relatively high proportion of FP with 21.7\%. The threshold values can be chosen differently in order to achieve a decreasing proportion of FP, which in turn leads to an increasing proportion of FN. This part of the algorithm is very flexible and can be adapted according to the purpose of its application.

The data of the 12 FN cases were inspected in order to further evaluate the performance of the method. It was found that in all cases the individuals did not have a nest and occupied a home range later in the summer. For the detection of the home range, only data during the months of February, March, April and May were considered. If data later in the year had been included, there would have been a risk of including movement data of birds after breeding. Later in the year, after breeding, there are birds that settle in a different place than the nest location and become stationary at communal roosting sites \parencite{spatz2022sex}. These behaviour patterns would be falsely identified by the algorithm as settlements in a home range. Later data could be of interest, especially considering that subadult individuals can settle temporarily in summer and autumn but do not breed yet. These temporary settlement areas are likely to be chosen as home range to start a breeding attempt in the spring of the following year \parencite{aebischer2021rotmilan}. These home ranges are ultimately also captured by the algorithm, as data from several years are analysed. Should it be of interest to specifically investigate cases of subadult settlements, the developed home range detection approach can be applied to subadult individuals by adjusting the time span of the input data.

The generated parameters are suitable for detecting a possible settlement in a home range, as the categories of cases with home range and without home range differ significantly both in the number of consecutive weeks with an area of less than 60~km$^2$ as well as in the distance of the centroid displacement (Figure~\ref{figure:hr_area_centroid_displacement_sex_diff}). Despite the significant differences in both parameters between the two categories (individuals with or without a home range), there was a high proportion of cases where a home range was falsely detected (FP). This is due to the fact that the main requirement was to miss as few settlements as possible. The increased proportion of FP was preferred over a higher rate of FN, as many of these cases can be filtered out in the later nest detection step.

An interesting finding is that the sex of the individuals has an influence on the number of consecutive weeks with an area of less than 60~km$^2$, where female red kites show a mean of 11.4 weeks compared to males with a mean of 12.6 weeks. Male red kites with a home range tend not to expand their area for a longer duration than female red kites. This behaviour could be due to the females expanding their home range as soon as they increase their contribution to the food supply for the nestlings, which happens when the nestlings are two to three weeks old \parencite{aebischer2021rotmilan}. For red kites breeding in western Switzerland, this moment is about 13-14 weeks after the start of nest building \parencite{scherler2023brutbiologie}. The expansion of the home range can be explained by the increasing food requirements of the nestlings \parencite{bischofberger2019werden, spatz2019raumnutzung}. The finding that females expand their home range towards the end of chick rearing and even exceed the values of the males is consistent with the findings of \textcite{spatz2019raumnutzung} and \textcite{spatz2021zwischen}. Apart from the fact that the sample size was small, \textcite{spatz2019raumnutzung} also emphasise the influence of brood loss, which in their study led to a shift in the home range area of females in particular. The reason why females occupy a home range of less than 60~km$^2$ for a significantly shorter duration than males cannot be conclusively explained. The potential influence of brood loss on the home range area was not analysed in this work. The fact that sex has no influence on the extent of the centroid displacement was to be expected given that the nest, as a fixed location, represents the focal point of the whole breeding cycle.



\subsection{Nest Detection}
With an accuracy of 88\%, the overall proportion of correctly predicted cases is high. The high recall of 99\% is achieved by the fact that only 1 out of 365 cases was falsely predicted to not have a nest. The precision (83\%) and the MCC (77\%) are slightly lower due to the fact that 11.7\% of the predicted cases were classified as FP. All statistical metrics have high values, indicating that nests can be reliably detected based on developed approach.

The data of the single false negative was inspected in order to further evaluate the performance
of the method. It was recognised that in this case there was too little data, which occurred too early in the year, so that the movement at the nest location was not emphasised enough. 

Both revisitations and residence time are suitable for detecting a possible nest location, as the categories of cases with a nest and without a nest differ significantly both in the mean daily revisitations as well as in the mean daily residence time (Figure~\ref{figure:res_time_revisits_sex_diff}). Despite the significant differences in both parameters between the two categories (individuals with or without a nest), there was a considerable proportion of cases where a nest was falsely detected (FP). A better separation was not possible as the focus was on detecting all nests and thus keeping the FN as small as possible while still aiming for an optimal separation of both categories. Should it be of interest to produce as little FP as possible at the cost of missing actual nests, the limit values of the mean daily revisitations and the mean daily residence time can be manually increased.

For red kites with a nest, there are significant differences between the sexes in both residence time at the nest and revisitation of the nest. On the one hand, it is obvious that females have a higher residence time at the nest, as the female takes care of incubation while the male provides food. On the other hand, the fact that revisitations are also higher for females than for males may seem counterintuitive. However, there are many occasions where the female leaves the nest for a short time and then returns. This may be to copulate, to defend the nest or to receive food from the male \parencite{aebischer2021rotmilan}. As the male is foraging most of the time and thus away from the nest location, fewer GPS locations are located at the nest, which consequently leads to a lower revisitation rate. The pattern may also be due to the fact that data were taken into account from February onwards, where the female was not yet breeding and the pair was preparing the nest.

For the nest detection method, limited prior knowledge of the species and the data structure regarding the revisitation rate and the residence time at the nest site must be available, but a seasonal restriction, as required in \textit{nestR}, is not needed. Furthermore, the nest detection method is based on recursive movement patterns, which is computationally more efficient than the Bayesian hierarchical model used in \textit{nestR}.



\subsection{Accuracy of Nest Locations}
The predicted nest locations correspond well with the actual locations according to the validation data (Figure \ref{figure:nest_dist_barplot}). With 95\% of all cases, most predicted nest locations are located within a radius of less than 120~m from the actual nest, which shows that the region of the nest can already reliably be localised.

The predicted nests with a distance of more than one kilometre to the actual nest location were all female individuals with an unsuccessful brood. The furthest distance of the predicted nest location to the actual nest location of a female with a successful brood is 243~m. For a male with a successful brood the furthest distance is 894~m.

The inaccuracies in locating the nest locations may be due to the fact that the brood history of the individuals was not taken into account. However, there were also cases where individuals built a nest but never got to breed. There were also cases of brood losses where individuals moved away from the nest after the loss and did not return to it. In these cases, the time in which the individuals showed a recursive movement pattern was limited, which made the detection of the nest difficult.

Statistically, there is no significant sex-specific difference in the performance of the detection of the nest location. It is interesting to see that nest locations of females and males can be determined with equal accuracy, especially considering the different movement patterns, which are most pronounced during the incubation phase. It appears that the nest site is the centre of attention for both females and males during the entire incubation period, which is clearly visible in the movement patterns. It can be concluded that there is no need for separate approaches for the detection of the nest site for both sexes.



\subsection{Brood Phase Identification}
\subsubsection{Daily Predictions}
The performance of the MLRM applied on the test dataset shows that for 64\% of the days it can be correctly predicted whether an individual has been incubating, feeding or non-breeding, while with 88\% the highest accuracy was achieved for non-breeding days. It can be concluded from this that the phase in which no breeding takes place can be very reliably separated from breeding phase. The other metrics (recall, precision and MCC) do not indicate an excellent performance. In addition, it was concluded from the low accuracy values for the categories incubating (49\%) and feeding (41\%) that they are hard to separate from each other. Therefore, another approach was to combine the two categories into the category breeding. This resulted in a high accuracy both overall (76\%) and for the two categories breeding (73\%) and non-breeding (78\%), while all other statistical metrics were also higher than with the raw model predictions approach. This has shown that GPS movement data can be successfully used to distinguish breeding days from non-breeding days.

As it would be desirable to predict the categories incubating and feeding separately, the third approach of rule-based adaptation was developed. This was intended to fill gaps where possible missing data or weather influences had affected the analysed movement patterns to such an extent that the actual behaviour could not be identified. This approach improved the predictions slightly compared to the raw model predictions. Different rule sets were developed and applied, some of which performed better, which was clearly due to over-manipulating of the predictions. It is a challenge to define rules that adjust incorrect predictions while not manually over-adjusting the predictions to meet the expectations. In this work, no rule set could be elaborated that improved the predictions considerably.

The performance of the algorithm is poor when applied to non-breeding individuals. All three approaches achieved similar predictions with an accuracy range of 41-56\%, which equals a random prediction. The fact that many days were classified as incubating or feeding shows that the behaviour of non-breeding individuals is partly comparable to that of breeding individuals, which can be explained by the fact that non-breeding individuals, especially young ones, often stay at common roosting sites that are revisited like a nest site \parencite{aebischer2021rotmilan}. The fact that this behaviour is misinterpreted in the predictions of the MLRM is not problematic, if the pattern of incubating and feeding days (Appendix~\ref{appendix:mlrm_predictions_nonbreeding}) could be detected by later classification and interpreted as a non-breeding individual.

When applied to the successfully breeding red kites in Thuringia, the performance of the algorithm was similar to the test dataset. The accuracies showed similar values, although not the non-breeding days showed the highest accuracies, but the breeding and feeding days. This was most evident in the merged categories approach, where 89\% of all breeding days were detected as such. Due to the different geographical regions and the associated different movement patterns, which are most evident in larger home ranges, this is a remarkable result, as it implies that the algorithm can be successfully applied to other geographical regions with different characteristics. An analysis of the performance of the home range and nest detection method on red kites in other geographical regions was not realised this work.

\subsubsection{Brood Success Classification}
The classification of predictions from the MLRM into successful and unsuccessful broods showed the best results when the merged categories approach was used as a basis. For the test dataset, 39\% of the cases were correctly classified as successful broods. For the red kites in Thuringia, even 86\% of the cases were correctly classified as successful broods. This result shows that the MLRM provides a good foundation for detecting successful broods. For the non-breeding individuals, 64\% of the cases were incorrectly classified as successful broods, which clearly shows the weaknesses of the approach of classifying the broods into successful and unsuccessful. However, this is not a weakness of the MLRM, but rather a weakness of the classification method, as the sum of days per category was added up. Considering that GPS telemetry-based datasets often show data gaps of one or several days, missing days can occur during the incubating or feeding phase. If the days of the respective category are added up, this number can be lower than the actual number of days. Therefore, an approach that considers the duration of the phases as a time span rather than the sum of all days might be more promising.

When the predictions of the MLRM are analysed visually, a rough pattern of a brood cycle corresponding to the expected pattern can be recognised for many cases from the predictions for the test dataset (Appendix~\ref{appendix:mlrm_predictions_test}) and the predictions for the individuals from Thuringia (Appendix~\ref{appendix:mlrm_predictions_thuringia}). In most cases, incubating days accumulate at the beginning of the breeding cycle, while feeding days accumulate in the end of the breeding phase. Overall, a brood can often be guessed from the pattern. For the non-breeding individuals, on the other hand, the plots show either much more noise or sections of only one of the categories incubating or feeding, resulting in no clear patterns (Appendix~\ref{appendix:mlrm_predictions_nonbreeding}). This implies that the predictions should be dividable into successful and unsuccessful broods based on the noise ratio in the predictions. 



\subsection{Research Objective}
An algorithm to quantify the breeding phenology of red kites was successfully developed, by implementing a set of suitable parameters to detect breeding-specific movement patterns. In a first step, it can be determined whether an individual has settled in a home range. In a second step, it can be determined whether it has a nest and where it is located. On a daily basis, it can then be successfully predicted whether an individual is incubating, feeding or non-breeding. Only for the final classification of whether an individual had a successful brood or not, no reliable solution could be found. The application of the algorithm for the daily prediction of breeding behaviour to red kites in Thuringia has proven the suitability of the algorithm for regions with different landscape characteristics and thus different movement patterns of red kites. The first two steps of the algorithm, home range detection and nest detection, could also be suitable for use in other geographical regions, as the thresholds of the metrics can easily be adapted to fit the movement behaviour of the population under study. In summary, the research objective was largely achieved, even if the final step of brood success could not be optimised to the desired extent. All sub-steps of the algorithm were successfully implemented and provided valuable insights in the breeding behaviour of red kites.