\subsection{Motivation}
To study population dynamics and the development of conservation strategies, it is essential to monitor and understand the movement behaviour of individuals as well as populations \parencite{allen2016linking, he2022guide, morales2010building}. For birds, this information can be used to derive the different behaviours, such as migration, foraging or breeding \parencite{picardi2020analysis}. An important characteristic that is crucial for population dynamics is the reproductive output, often characterized by the breeding success. In order to monitor and evaluate the development of a bird population over time, the determination of the breeding success of individual breeding pairs plays a key role, as this helps to understand the population's reproductive output and the underlying mechanisms shaping it. Today, this information is mostly collected in the field through nest observations or cameras attached to the nest \parencite{scherler2023brutbiologie}.

Since manual data collection in the field is very labour- and personnel-intensive, it is of great advantage to remotely analyse behavioural movement patterns by means of GPS telemetry \parencite{kays2015terrestrial, williams2020optimizing}. Therefore, in recent years, intensive work has been conducted in the research field of movement ecology on developing methods to remotely determine various aspects of a bird's life \parencite{bowgen2022curves, berger2015recursive, bracis2018revisit, gurarie2016what, joo2020navigating, kane2022understanding, morales2010building, picardi2020analysis, schreven2021nesting}. Different approaches allow it to understand ecological mechanisms such as migration behaviour \parencite{garcia2022seasonal, maciorowski2019autumn}, breeding behaviour \parencite{picardi2020analysis} or foraging behaviour \parencite{bracis2018revisit} from movement data and to interpret how they interact with environmental variables.

A method based on revisitations of specific locations in movement tracks allows to detect ecologically important sites, such as the nest location of a bird \parencite[, R package \textit{recurse}]{bracis2018revisit}. \textcite{picardi2020analysis} have created an R package called \textit{nestR}, which is designed to find a bird's nest and determine breeding success. After defining species-specific parameters, namely the beginning and end of the season and the duration of the nesting cycle, a Bayesian hierarchical model is used to analyse the recursions in a bird's movement trajectory and predict whether a brood was successful or not. While \textit{nestR} accounts for certain species-specific parameters, it does not account for changes in movement behaviour between individuals and between different phases of the brood. Despite the parameters that can be defined, there are additional intrinsic and extrinsic factors that can influence the movement behaviour of a bird during breeding. The sex of a bird can play an important role, as the movement behaviour of the two sexes can differ significantly due to different brood-specific tasks \parencite{spatz2021zwischen, spatz2022sex}. Also, there are different phases during a breeding cycle, which is why it cannot be assumed that the movement behaviour is constant over the entire breeding cycle \parencite{spatz2019raumnutzung}. Furthermore, the assumption that a bird only stays at the nest during the breeding cycle may be wrong, as a bird can stay at the nest both before and after the breeding phase \parencite{aebischer2021rotmilan}.

The above-mentioned parameters (sex, brood phase and time of leaving the nest) apply to the behaviour of the red kite (\textit{Milvus milvus}). There is a clear division of roles during breeding, in which the female bird is mainly responsible for incubating the eggs, while the male bird is responsible for providing food \parencite{spatz2021zwischen, spatz2022sex}. The brood phase also influences the behaviour of red kites \parencite{spatz2019raumnutzung}. For example, the female bird mainly stays at the nest location during the incubation of the eggs, while she also forages during the feeding of the nestlings \parencite{aebischer2021rotmilan, spatz2021zwischen}. There are many records of red kites spending long periods of time at the nest location after successfully rearing their young \parencite{paquet2015premiers, spatz2021zwischen}. 

Until now, there is no algorithm that can capture the breeding cycle of a bird while allowing a probabilistic determination of the different breeding phases. There is also no solution that allows for a sufficiently broad selection of breeding-relevant factors while calculating important parameters such as home range sizes, also accounting for differences in spatial behaviour patterns of different birds and locations.

\subsection{Research Objective}

The aim of this work is to develop an approach to study the breeding cycle of red kites, while taking into account important factors that can significantly influence movement behaviour, such as sex and the different phases of a brood. An algorithm is developed that allows detailed fine-tuning to account for the heterogeneity of behavioural patterns between individuals and brood phases. In this master’s thesis the breeding behaviour of the red kite is analysed based on GPS movement data collected by the Swiss Ornithological Institute. The following research objective can be formulated:

\vspace{1\baselineskip}

\begin{center}
    \textit{An algorithm is to be developed that allows to quantify the breeding phenology of red kites based on GPS movement data.}
\end{center}

\vspace{1\baselineskip}

The expectation is to find movement-specific parameters that can be successfully applied to analyse the movement behaviour of red kites. There are various parameters that have been developed and successfully applied in movement ecology for the analysis of movement behaviour. Movement parameters characterising ranging and breeding behaviour will be applied in an integrated workflow that enables the detection of specific phases and events throughout the breeding cycle. In differnet steps, it will be determined whether a bird has settled in a home range and whether it has a nest. Furthermore, the two phases of incubating the eggs and feeding the nestlings will be determined separately. In a last step, the breeding success will be derived from the respective duration of the two phases.

Finally, it will be examined whether the developed approach can be successfully transferred to regions outside the study area. The algorithm for the data collected by the Swiss Ornithological Institute will be applied to another population of red kites in Thuringia, Germany, to evaluate the performance of the algorithm in a different habitat. The ultimate goal is to develop a reliable algorithm applicable to all red kite subpopulations throughout the distribution range.