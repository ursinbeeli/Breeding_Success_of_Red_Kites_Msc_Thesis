Several people accompanied me during the process of writing my master's thesis and supported its successful completion.

First of all, I would like to thank my two supervisors, Robert Weibel from the Department of Geography at the University of Zurich and Patrick Scherler from the Ecological Research Unit at the Swiss Ornithological Institute in Sempach. I would like to thank Robert Weibel in particular for his active support in the process of finding a suitable topic as well as in following support during several meetings. I am very grateful to Patrick Scherler for his energetic support in creating a functioning and biologically meaningful algorithm in several, also unscheduled, meetings. I also owe him moral support, as he was able to motivate me and to show me more promising paths in several dead ends.

A big thank you goes to the Swiss Ornithological Institute for making it possible for me to write my master's thesis in such an interesting field of research and for trusting me to work with the huge GPS movement dataset of red kites.

A special thanks goes to Thomas Pfeiffer who provided me with a GPS movement dataset of red kites in Thuringia, Germany, which I was allowed to use as an evaluation dataset for my algorithm.

I would also like to thank the GIScience Unit at the University of Zurich, which gave me constructive feedback during several presentations. The same goes for the Ecological Research Unit at the Swiss Ornithological Institute in Sempach, where I would especially like to thank Florian Orgeret for his proactive support.

Last but not least, I would like to thank my friends and family who have been understanding with my limited capacity and supported me in this process. In particular, I would like to thank Nina Kohout, who patiently offered me moral support throughout the entire time.